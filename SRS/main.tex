% main.tex
% https://www.overleaf.com/8524532369xzdjrdytffyx#f690f0

\documentclass[a4paper,12pt]{article}

\usepackage[margin=1in]{geometry}
\usepackage{titlesec}
\usepackage{fancyhdr}
\usepackage{graphicx}
\usepackage{hyperref}
\usepackage{longtable}
\usepackage{multirow}
\usepackage{amsmath}
\usepackage{cite}


\pagestyle{fancy}
\fancyhf{}
\lhead{\includegraphics[width=0.05\textwidth]{assets/college-logo.png}}
\rhead{\textit{SRS Document}}
\lfoot{\includegraphics[width=0.05\textwidth]{assets/logo-beta.jpg}\textit{ Hukum}}
\rfoot{\thepage}

\title{
    \begin{figure}[h!]
        \centering
        \raisebox{0.2cm}{\includegraphics[height=2cm]{assets/college.png}}
        \hspace{1cm}
        \includegraphics[height=2cm]{assets/logo-beta.jpg}
    \end{figure}

    \large \textbf{Indian Institute of Information Technology, Dharwad}\\
    \textbf{Software Requirement and Technical Feasibility Document}\\
    \vspace{0.5cm}
    \large \textbf{\textit{CS310}}: Software Engineering Project
}

\author{
    Prepared by: Team \textbf{Hukum}\\
    Group Members: \\[1em]
    \begin{tabular}{ll}
        Amritanshu Aditya & \textit{23bcs013} \\
        Barghav Abhilash B R & \textit{23bcs028} \\
        Dhruv A. Koli & \textit{23bcs044} \\
        K V Modak & \textit{23bcs067} \\
    \end{tabular}\\[2em]
    Faculty Advisor: \\
    Dr. Vivekraj V K\thanks{Dept. of Computer Science, IIIT Dharwad}\\ \\
    note\thanks{\textbf{\textit{This is not the official documentation of our product.}}}
}

\begin{document}

\maketitle

\newpage
\tableofcontents
\newpage

\section{Introduction}
\subsection{The Problem Statement}
\subsubsection{The Traditional Attendance Procedure}

\begin{itemize}
    \item Current methods are time-consuming and prone to human error.
    \item Manual entry increases the administrative workload for faculty.
\end{itemize}

\subsubsection{Hardware Dependency}

\begin{itemize}
    \item Existing automated systems often require expensive hardware, such as biometric devices or card readers.
    \item High hardware costs make these systems less accessible to many institutions.
\end{itemize}

\subsubsection{Proxy Attendance}

\begin{itemize}
    \item Traditional systems, both manual and some automated, allow for proxy attendance, compromising the accuracy of attendance records.
\end{itemize}

\subsubsection{Data Management Challenges}

\begin{itemize}
    \item Manually maintaining attendance records can lead to inconsistency and data loss.
    \item Retrieval and analysis of attendance data are cumbersome in traditional systems.
\end{itemize}

\subsubsection{Scalability Issues}

\begin{itemize}
    \item Many current systems are not scalable to accommodate larger class sizes or multiple sessions without additional resources.
\end{itemize}

\subsubsection{User Experience}

\begin{itemize}
    \item Complexity in using some existing systems can deter users, reducing system adoption and efficiency.
    \item Maintenance of these systems can also be challenging and resource-intensive.
\end{itemize}

\subsection{Our Approach}

We solve this issue by leveraging image recognition for identifying the students for attendance.
We also involved automation of attendance marking system to reduce manual work for all parties involved in this.

\subsection{What the customer wants?}

\subsubsection{Cost-effective Attendance System}

\begin{itemize}
    \item A system designed to minimize expenses while efficiently marking attendance.
\end{itemize}

\subsubsection{Reduced Time for Marking Attendance}

\begin{itemize}
    \item The purpose of this project is to streamline and expedite the attendance process, saving valuable time.
\end{itemize}

\subsubsection{Minimized Proxies}

\begin{itemize}
    \item Implement measures to significantly reduce the incidence of proxy attendance.
\end{itemize}

\subsubsection{Ease of Use and Better Maintainability}

\begin{itemize}
    \item Ensure that the system is user-friendly and easy to maintain, promoting long-term usability.
\end{itemize}

\subsection{Objectives}

Our approach to creating an efficient attendance marking system is clear and practical. The main objectives are as follows.

\subsubsection{Cost Efficiency}

\begin{itemize}
    \item Most automated attendance systems require significant hardware investments.
    \item Our model provides a cost-efficient solution that is entirely software-based.
    \item Using image recognition and automation, we eliminate hardware costs by leveraging open-source libraries to achieve the desired results.
\end{itemize}

\subsubsection{Minimized Proxies}

\begin{itemize}
    \item Both manual and automated systems often have a margin of error, allowing proxies.
    \item Our solution addresses this by requiring students to physically be present in the classroom.
    \item Faculty members take images as evidence, which are used for image recognition.
    \item This ensures that proxies are nearly impossible, as the process depends entirely on the faculty's timing to capture and upload images.
\end{itemize}

\subsubsection{Automation}

\begin{itemize}
    \item Image recognition is the foundation of our system.
    \item The system automatically generates attendance records for sessions conducted by faculty.
    \item All attendance data is logged into a web portal accessible to administrators, students, and faculty.
\end{itemize}

\subsubsection{Ease of Use and Maintenance}

\begin{itemize}
    \item Our system is designed to be user-friendly and maintainable.
    \item The web portal provides easy access to all required data for clients, students, and faculty, ensuring convenience and compliance.
\end{itemize}

\subsubsection{Scalability}

\begin{itemize}
    \item The system is scalable to accommodate the entire class of 70 students.
    \item Tests have shown that our image recognition model can accurately identify faces with over 66\%\ accuracy for a single column of students in the classroom. Check references.
\end{itemize}

\section{Software Requirements}
\subsection{Basic Requirements}
\subsubsection{Scope}
\begin{itemize}
    \item The software should be such that it allows the instructor of the course to access and manage the attendance of all students in their class. 
    \item The software should allow students to see which classes they have been marked as present and which classes they have missed for all subjects in their particular semester.
    \item After marking attendance for a class, the software should have a feedback system for the students that would let them know that they have been marked present or absent in the attendance database.
    \item The software should have great precision in recognizing students and should have minimal or no mismatches.
    \item In the rare case of mis-detection, the software should allow the course instructor to edit and update attendance for the missed person, provided that the aggrieved student approaches them within a reasonable time.
\end{itemize}

\subsubsection{Budget}
\begin{itemize}
    \item The project requires minimal investment. The training database will consist of photos of the students, which the software can map to photos taken in class for attendance marking. 
    \item Photos that will be uploaded daily will be stored on an on-line server that handles File Transfer Protocol (FTP). The database will request an API from this server to access those images and mark attendance for the students.
\end{itemize}

\subsubsection{Time}
\begin{itemize}
    \item Data Training and Data Collection: 3 to 4 weeks
    \item Development: 8 to 9 weeks
    \item Testing (Phase-wise testing and final testing): 2 to 3 weeks
\end{itemize}

\subsection{HLD}
\begin{figure}[h]
    \centering
    \includegraphics[width=\textwidth, keepaspectratio]{assets/HLD_version19jan.excalidraw.png}
    \caption{HLD}
    \label{fig:hld_diagram}
\end{figure}


\newpage

\section{Technical Feasibility}
\subsection{Feasibility Analysis}
\subsubsection{Data Collection and Training}
\begin{itemize}
    \item Collection of adequate face data for all students.
    \item Training the face recognition model to recognize on the basis of the training data.
    \item Testing the system using varying light and occlusions.
    \item Can be done using pre-trained models such as \textbf{FaceNet} and \textbf{DeepFace} as well.
\end{itemize}

\subsubsection{Platform for customer}
\begin{itemize}
    \item A web platform where user provides photos as input.
    \item Photos can be uploaded or taken live. However, the one with easier functionality for the user and the one that is less error-prone may be implemented.
    \item The photos are taken column-wise, ensuring that everyone's face is properly visible.
    \item One class will generate an input of 4 photos, 1 of each column. 
\end{itemize}

\subsubsection{Processing of Images}
\begin{itemize}
    \item Includes \begin{itemize}
        \item Face Detection
        \item Face Recognition
        \item Mapping of records
    \end{itemize}
    \item The images are processed via the backend which is trained upon the training image data.
\end{itemize}

\subsubsection{Feedback}
\begin{itemize}
    \item Students receive an email notification regarding their attendance status for the class.
    \item In any rare case of face mismatch, manual overriding of records is possible.
\end{itemize}

\subsubsection{Monitoring}
\begin{itemize}
    \item The attendance database are constantly monitored by our developers who oath to work honestly in the surveillance of the records.
    \item Helps reduce any potential discrepancies and aids in the further scalability of the product.
\end{itemize}

\subsection{Cost \& Benefit Analysis}
\subsubsection{Cost Analysis}
\begin{itemize}
    \item Storage requirements for new photos.
    \item Backup photos for a period of two weeks in case of discrepancies.
    \item A working camera phone.
\end{itemize}
\subsubsection{Benefit Analysis}
\begin{itemize}
    \item Time and Cost efficient.
    \item Almost negligible proxies.
    \item More hassle free for students and faculty.
    \item Students get the access to their attendance records 24/7 anywhere from the world.
\end{itemize}

\subsection{Technology Stack}

\begin{table}[h]
\centering
\renewcommand{\arraystretch}{1.5}
\begin{tabular}{|p{0.2\textwidth}|p{0.35\textwidth}|p{0.35\textwidth}|}
\hline
\textbf{Component} & \textbf{Primary Options} & \textbf{Alternative Options} \\
\hline\hline

\multirow{1}{*}{\textit{Frontend}} & 
React.js with TS / JS & 
Vue.js with TS \& Vuetify \\
\hline

\multirow{1}{*}{\textit{UI Components}} & 
Shadcn/ui, Tailwind CSS & 
UnoCSS \\
\hline

\multirow{1}{*}{\textit{Backend API}} & 
FastAPI, Python 3.9+ & 
Django REST, Flask \\
\hline

\multirow{1}{*}{\textit{Face Recognition}} & 
OpenCV, face\_recognition & 
DeepFace, MediaPipe(Google) \\
\hline

\multirow{1}{*}{\textit{Database}} & 
PostgreSQL, SQLite (for local testing) & 
Redis (Caching/Session), MongoDB \\
\hline

\multirow{1}{*}{\textit{Storage}} & 
Google Drive API, Local Storage & 
Amazon S3 (free tier) \\
\hline

\multirow{1}{*}{\textit{Email Service}} & 
SMTP (Python smtplib) & SendGrid (free tier) \\
\hline

\multirow{1}{*}{\textit{Testing}} &
pytest, integration tests & ||||||||||||| \\
\hline

\multirow{1}{*}{\textit{Dev Tooling}} &
DevLogs, Documentation, GIT VCS & Code Linting \\
\hline

\multirow{1}{*}{\textit{Deployment}} & 
Heroku (free tier), Vercel & 
Docker, Netlify \\
\hline

\multirow{1}{*}{\textit{Monitoring}} & 
Basic Python logging & 
Sentry (for error tracking) \\
\hline
\end{tabular}
\end{table}

\newpage

\begin{thebibliography}{9}

\bibitem{opencv} 
Bradski, G., \textit{The OpenCV Library}, Dr. Dobb's Journal of Software Tools, 2000. \url{https://opencv.org/}.

\bibitem{face_recognition_python} 
Adam Geitgey, \textit{Face Recognition with Python}, 2018, \url{https://github.com/ageitgey/face_recognition}.

\bibitem{yt}
SmartPencs, \textit{Attendance Management System Using Face Recognition} Live Face Detection without face recognition \url{https://rb.gy/kelprs}.

\bibitem{HLD}
Excalidraw, \textit{High Level Design for Project} Whiteboard tool that lets you easily sketch diagrams
\url{https://excalidraw.com} \& \url{https://libraries.excalidraw.com/}.

\bibitem{FTP}
File Transfer Protocol, Useful in handling file (image) receive and send operations between server and face recognition code.
\url{https://en.wikipedia.org/wiki/File_Transfer_Protocol}

\bibitem{Google Drive API}
Google Drive API, \textit{Google Drive API for file upload and downloads} 

\bibitem{benchmark} 
OpenCV claims a successful match of faces with a score more than 66\%.

\end{thebibliography}

\vspace{400pt}
\textit{A document created using} \LaTeX ||||||||||||||| \date{\today}
 
\end{document}

